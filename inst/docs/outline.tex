\documentclass{article}
\usepackage{times}
\usepackage{fullpage}
\usepackage{comment}

% These come from the Docs/ directory in org/omegahat.
\input{WebMacros}
\input{SMacros}

\def\MatlabFun#1{\textsl{#1}}

\title{R-Matlab Interface}
\author{Bitao Liu\\
Duncan Temple Lang\footnote{Please send correspondence 
to duncan@wald.ucdavis.edu.}
}
\begin{document}
\maketitle

\begin{abstract}
This is a \textit{very early draft} of documentation discussing
the RMatlab interface.


  Matlab and R are two interactive, high-level programming languages
  used in scientific computing.  The languages have a lot in common
  but have very different target audiences and focii.
  R  is primarily used by the statistical community for advanced data analysis
  and research in statistical methodology; Matlab is primarily used  
  by engineers. The functionality of the two languages does overlap,
  but also complement each other very well. R has little suppport for
  image processing, differential equations, and so on,
  while Matlab has only limited support for modern statistical
  methodology and stochastic data analysis. Of course, both languages provide the primitives 
  to develop such facilities, being general purpose programming
  languages, but it makes much more sense to reuse existing,
  specialized code when it is available. Rather than requiring 
  Matlab users to program in R to access statistical functionality,
  and R users to write Matlab code for engineering-oriented tasks, 
  we present an interface that allows one to call R from Matlab and
  vice versa.
  This allows each group 
  to transparently leverage the functionality provided by the other.
  We describe the \SPackage{RMatlab} package which provides this inter-system
  interface. The model is quite simple and general and mimics that
  used in other Omegahat inter-system packages (see SJava, RSPerl, 
   RSPython, ROctave, RXLisp, RinS). 
\end{abstract}


\section{Introduction}
As we mention in the abstract, Matlab and R share many similarities,
but are quite different in many ways.  The S language -- of which R is
a dialect -- is quite rich both in terms of statistical data analysis
and numerical computation, and also software development.  The R
environment provides advanced conceptual but common facilities used by
statisticians in research and data analsysis.  It allows statisticians
to think about computations in much the same way as they do the
underlying statistical methodology and data analysis. Records, data
frames, subsetting, formulae, models are aspects that make R uniquely
statistical in nature.  Matlab is significantly more oriented towards
interactive engineering tasks. The language is more foucsed on direct
matrix calculations and providing the primitives for numerical
computing.  That language provides the essential data structures,
adding to its simplicity, sometimes at the expense of its
expressiveness.  Matlab focuses on computations
such as eigen-value.


One of the motivations of this package (and other inter-system
packages) is to allow scientists to program their computational tasks
quickly and easily and using well-tested and reliable software.  When
possible, reusing existing software that has been used extensively by
others is significantly better than writing our own.  And avoiding the
overhead of writing such software and distracting ourselves from the
original task allows us to focus on expressing a computation directly
and succinctly.


\section{The Interface}
The \SPackage{RMatlab} package provides a way for R to call Matlab
functions, and Matlab to call R functions.  We can start Matlab from
within R, or we can embed R within Matlab.  When we do the latter
(starting R within Matlab) using the \SPackage{RMatlab} package, we
start an R interpreter as MEX-accessed C code within the Matlab
process.  This allows us to call R functions from within the Matlab
process using the same address space.  This makes the inter-system
communication very fast and allows objects to be shared directly by
reference.  And since R is embedded within Matlab, it can 
make calls back to Matlab.  This allows for a range of intresting
computations, and ensures that we can use the two languages
in tandem and program in the most convenient environment for
individual tasks.

Calling R from Matlab can occur in two ways.  As we have just
discussed, if R is started from within a Matlab session, R functions
invoked from Matlab can call back to Matlab functions.  This is called
the MEX interface.  The two interpreters share the same single process
space and the communication is direct in the same way that Matlab
calls MEX files and other external interfaces.

The second way that R can access Matlab is by starting a separate
Matlab process and sending commands to it.  This is the Engine
Interface to Matlab. (This same mechanism can be used for connecting
Matlab to R on Windows using the DCOM server interfaces provided by
R.)  The fact that this is a separate process makes the communication
slower but allows us to have the process on a different machine. This
allows us to do a form of distributed computing.  Unfortunately, the
Engine API provides only a limited set of routines by which we can
access the Matlab interpreter.  They suffice to allow us to hide the
differences between the MEX and Engine interface, but the
implementation details do have consequences for the Engine interface.
We will discuss these below.





\section{Calling R from Matlab}

We provide 3 simple functions in Matlab to access functionality in R.
The first is used to initialize the R engine and the other two are
used to invoke arbitrary R functions.  The functions are
\MatlabFun{initializeR} \MatlabFun{callR} and \Matlab{callNamedR}.
Using these, one has access to most all of R.

\subsection{initializeR}
It seems natural to expect that one must start the R interpreter
before calling any R functions.  And we do this from within Matlab
using the function \MatlabFun{initializeR}.  (The package needs to
ensure that this is done before calling an R function.)  The inputs to
\MatlabFun{initializeR} should consist of a Matlab ``cell'' giving the
command line arguments to R as strings.  We note that the first
element of these command line arguments should be a name to use for
the program; we use \texttt{RMatlab}. It currently has no consequence.
The remaining arguments are the regular ones that we would provide
to R on the command line when starting R from a shell. 

An example will make this more concrete.
The following creates an instance of the R interpreter
with the Matlab process.
\begin{verbatim}
initializeR({'Rmatlab' '--no-save' '--no-restore'})
\end{verbatim}
This tells R not to read an .RData file, if it exists, in the working
directory on startup and to not save the global environment/workspace
when terminating R.

Of course, we can create the argument cell structure in a separate
step and store them in a Matlab variable and then pass them to the
\MatlabFun{initializeR} call.  For example,
\begin{verbatim}
args = {'Rmatlab' '--no-save' '--no-restore'}
\end{verbatim}
or
\begin{verbatim}
args = cellstr(char('Rmatlab', '--vanilla'))
\end{verbatim}
creates the variable \texttt{args}.  Then, we can call thes
initialization routine to start R with these arguments.
\begin{verbatim}
initializeR(args)
\end{verbatim}

The reason for using cells rather than strings is to avoid padding the
string elements with blanks.  In Matlab,
\begin{verbatim}
 [ 'RMatlab' ; '--no-save']
\end{verbatim}
gives an error because the strings don't have the same length.
Unfortunately, Matlab stores its string ``vectors''
as a matrix and so each string (i.e. element in the vector)
must have the same number of characters.
The function \MatlabFun{char} 
avoids the error and is a convenient way to create
a Matlab object storing the strings:
\begin{verbatim}
 char('RMatlab', '--no-save')
\end{verbatim}
Again, this creates a matrix whose dimensions are 1 by 8; in other
words, the RMatlab is extended with an extra space to make it the same
length as \texttt{--no-save}.  Accordingly, the inputs cannot be used
directly as command line arguments and we would have to trim the
arguments.  The result is that we use a cell array to represent the
command line arguments as is, without the extra padding.


Note that we can easily define a M-function in Matlab to provide a
more convenient interface to this \MatlabFun{initializeR} primitive.
The function
\begin{verbatim}
function status = startR(varargin{:})
  status = initializeR([ {'RMatlab'}, varargin ] )
end
\end{verbatim}
merely avoids the need to remember to specify
the \texttt{RMatlab} as the first command line argument.
This is a small advantage!


\subsection{callR}
The next and most important function/routine in the Matlab-to-R is
\MatlabFun{callR}.  This is the mechanism with which we can invoke an
R function from within Matlab.  This is the fundamental building block
with which we can gain access to almost everything in R.

The function is quite straightforward.  The first argument is the name
of the R function to be called and this is required.  The remaining
arguments are the values to be passed (in the same order) as arguments
to the R function.  These arguments are marshalled to the R engine
using the conversion mechanism discussed below. 
The result of the R function is converted back to a Matlab
value and returned as the result of the 
\MatlabFun{callR} function.


Let's start with a very simple example of generating random numbers
using R's RNG facilities.  We generate 10 random values from the
Poisson distribution with mean 3 using the call
\begin{verbatim}
o = callR('rpois', 10, 3)
\end{verbatim}
This is equivalent to evaluating the R expression
\begin{verbatim}
 rpois(10, 3)
\end{verbatim}
The values 10 and 3 are converted to the corresponding values in R
(numeric vectors of length 1) and then passed in the call to
\SFunction{rpois}.  The resulting R integer vector of length 10
containing the random numbers is converted to a 10 by 1 Matlab array
and this is assigned to the variable \texttt{o}.  We can use this
directly within Matlab in subsequent commands without any reference to
R and where the data come from.  For example,
\begin{verbatim}
mean(o)
\end{verbatim}


Arbitrary R functions can be called using \MatlabFun{callR} using the
name of the function.  Clearly, all that is needed is that the Matlab
values are marhsalled across to the R call and the result returned
appropriately.  Soon, we will be able to invoke anonymous functions
that are returned as results of earlier calls.


\subsection{Named Arguments in R Function Calls}

The function \SFunction{rnorm}, for example, illustrates a problem
with the \MatlabFun{callR}.  Suppose we want to generate numbers from
a $N(0, 9)$ distribution, i.e. with an SD of $9$.  In R, we can use
named arguments as in
\begin{verbatim}
 rnorm(10, sd = 9)
\end{verbatim}
This allows us to omit the value of the mean and use the default value
for that argument.  In other R functions, the fact that an argument is
missing rather than specified with the default value can be important.
So we need to be able to call R functions with explicit names for the
arguments.  Unfortunately, given my current and limited knowledge of
Matlab, there are no named arguments in Matlab. Instead, people use
conventions to associate names with values.  Without this as a
fundamental part of the language, the conventions become akward.
For example, suppose we want to call an R function
\begin{verbatim}
bar =
function(x, y, xlab = "X", ylab = "Y", title = "A Plot")
{
  ...
}
\end{verbatim}
and pass it vectors for the first two parameters
and a value for the \SArg{title} parameter.
We could use something like the following:
\begin{verbatim}
 callR('bar', 
        'x', [1 2 3 4], 'y', [10, 9, 8, 7], 'title', 'My plot')
\end{verbatim}
Here we identify \textit{each} argument by name.  This is not a bad
thing as it makes the call explicit and avoids any ambiguity.  The
downside is that we \textit{must} specify the arguments even when the
position would make some of them clear. 

In R, we would specify this call as
\begin{verbatim}
bar( c(1, 2, 3, 4), c(10, 9, 8, 7), title = 'My plot')
\end{verbatim}
We can leverage this approach in the Matlab interface using a differnt
function, say \MatlabFun{callNamedR} in which we specify the arguments
as
\begin{verbatim}
callNamedR('bar',
           [1 2 3 4], [10, 9, 8, 7], 
           {'title', 'My plot'})
\end{verbatim}
The idea is that unnamed arguments are given as regular arguments and
the named arguments are given as a Matlab cell consisting of
name-value pairs.

To call \SFunction{rnorm}  in our initial example above,
we can now use
\begin{verbatim}
callNamedR('rnorm', 10, {'sd', 9})
\end{verbatim}


\subsection{Evaluating R commands}
Many users who think about calling R from Matlab or vice verse will be
inclined to send an R command from Matlab.  If we want to generate 10
values from a $N(0, 9)$ as above, we could evaluate the command
\begin{verbatim}
 rnorm(10, sd = 9) 
\end{verbatim}
We have provided an R function named
\SFunction{.REvalString} in the RMatlab package
which can be used to evaluate R commands all parcelled up
as a string.
The result is returned directly to the  caller
and so this can be called in Matlab as 
\begin{verbatim}
data = callR('.REvalString', 'rnorm(10, sd = 9)')
\end{verbatim}

This is convenient for simple use.  Unfortunately, it does not work
well for more general situations.  We have to construct the
command as a string. When we can type the inputs as a literal
string (as above), this is fine. 
When we have to merge values that are stored in variables, this
becomes
more tedious.  For example, if we have a Matlab function
that takes the mean and sd as arguments, we have to
concatenate the values into the command string.
This might be done with the Matlab command
\begin{verbatim}
 strcat('rnorm(', num2str(n), ', sd = ', num2str(sd), ')')
\end{verbatim}
This is certainly less elegant than a direct call to
\SFunction{rnorm}.
It gets worse in two other ways.
If we allow different types for the inputs
(e.g. numbers, strings, cells, etc.), we will need a more general
mechanism to convert the values to strings.
And secondly, we have to be careful with quotation marks in strings.
We must escape quotes within the command.



\section{Basic Examples}




\section{Advanced Examples}
Ode.

Fast matrix operations.

Graphics.

Round-tripping to ensure that we get what we expect.

x = callR(".MatlabMexCall", "matrix", 2, 2)



\section{Calling Matlab from R}

Calling Matlab from R turns out to be a little more limited than the
other way around.  This is primarily because the folks at MathWorks
have apparently provided a more traditional model for interacting with
the Matlab interpreter or ``Engine''.  This model allows only for
evaluating Matlab commands presented as strings and setting and
retrieving Matlab variables in the work space.  We can use this to
implement a reasonably powerful connection between R and Matlab, but

it is somewhat limited and more akward than being able to call Matlab
functions directly and anonymously.

The \SPackage{RMatlab} provides functions
for  the following.
\begin{itemize}
\item[.MatlabInit] starting the Matlab engine,
\item[.MatlabEval] evaluating a Matlab command,
\item[.MatlabGet] retrieving the value of a
variable in the Matlab  workspace,
\item[.MatlabPut] setting the value of a
variable in the Matlab workspace.
\item[.MatalClose] closing the  Matlab engine.
\end{itemize}

These primitives allow us to construct Matlab commands in R and to
fetch the results.  We must start by initializing the Matlab engine.
We can pass different command line options to the startup just as we
would when invking Matlab from the Unix shell.  (On Windows, these are
ignored.)  Rather than invoking the matlab executable, we specify
$"\\0"$ as documented in the Matlab manual\cite{MatlabExternalManual}.

With the engine initialized, we can evaluate Matlab commands.

Our model when calling R from Matlab was to 
call a function by name and pass the values directly from Matlab to R,
e.g.
\begin{verbatim}
 x = callR("foo", 1:10, {"a", "bcde"}, rand(4, 3))
\end{verbatim}
If we had a reference to an anonymous (un-named) R function, then
we would be able to invoke that also with the same mechanism.
The primitives above do not allow us to use the same
model directly. However, with a few drawbacks, we can hide
the details from the user.
We would like to be able to call a Matlab function from
R in much the same way  as above, something like
\begin{verbatim}
x = .Matlab("foo", 1:10, c("a", "bcde"), matrix(rnorm(12), 4, 3))
\end{verbatim}

If we were to use the \SFunction{.MatlabEval} function directly,
we would have to write each of the arguments in
its Matlab format with the values currently in the R arguments.
This would be something like
\begin{verbatim}
"foo([1 2 3 4 5 6 7 8 9 10], {\"a\", \"bcde\"}, [
\end{verbatim}

In other words, we have to serialize each R object in a form suitable
for constructing the corresponding value in Matlab.
For a matrix, we would need a function something like
\begin{verbatim}
paste("[", paste(apply(m, 1, paste, collapse = " "), collapse = "; "), "]")
\end{verbatim}
Of course, this string representation loses precision in the numbers
by limiting the decimal places, although we can control this.  And we
have to be careful to escape the quotes for character matrices.  All
in all, this is quite ugly and not very general.  What about
transferring R functions as arguments, or a model formula, or a
connection object.  Well of course, these don't have a corresponding
form in Matlab that is useful.  But we often want to pass these in a
function call as arguments to another function call which might be in
R.

Armed with the \SFunction{.MatlabPut} and \SFunction{.MatlabGet}
functions, we can provide a more flexible interface that can
potentially grow to accomodate a richer interface.  Rather than
constructing a Matlab command by serializing the R arguments as
strings, we can first marshall each value to Matlab and assign it to a
variable in the Matlab workspace using \SFunction{.MatlabPut}.  Then,
we can construct a simpler Matlab command to call the function of
interest referencing these named variables and assign the result to a
particular Matlab variable.  Then, we can marshal that value back to R
via the \SFunction{.MatlabGet} function.  Our \SFunction{.Matlab}
function is now more\footnote{but still not actually elegant}
elegantly written as 
\begin{verbatim}
.Matlab = 
function(funcName, ..., engine)
{
  args <- list(...)
  names(args) <- paste("r_arg", 1:length(args), sep = "_")
  .MatlabPut(.values = args, engine = engine)

  on.exit(.MatlabRemove(names(args), engine = engine))

  cmd = paste("r_result = ", funcName, "(", 
                  paste(names(args), collapse=", "),
               ");")
  .MatlabEval(cmd, engine = engine)

  ans = .MatlabGet("r_result", engine = engine)

  .MatlabRemove("r_result", engine = engine)
  ans
}
\end{verbatim}
\footnote{The version in the R package is slightly more
sophisticated than this, but only in dealing with
names of R arguments.}

What is missing in this interface?
The ability to use the mexCallMATLAB
would be useful.
The implementation above uses global variables
within the Matlab workspace. This could cause problems.

It might be possible to be able
to call mexCallMATLAB in the restricted
case of calling Matlab from R from within
Matlab. In other words, this may require
that we initiate  the original communication from
Matlab but callback to Matlab from R.

The idea is quite simple.
From within Matlab, we start R.
Then we call R to load the
RMatlab package to load
the R functions to access Matlab 
and the associated C code in RMatlab.so:
\begin{verbatim}
callR('library', 'RMatlab')
\end{verbatim}
Then we can use the following
\begin{verbatim}
o = callR('.MatlabMexCall', 'rand', 1, 1)
\end{verbatim}
to a very basic computation.  We start from Matlab and call the R
function \SFunction{.MatlabMexCall}.  We pass it 3 argument - the name
of a Matlab function to call and arguments ($1$ \& $1$) to be passed
to that Matlab function.  R evaluates the call to
\SFunction{.MatlabMexCall} and accordingly invokes the specified
Matlab function 'rand'.  Matlab returns the answer to R and R returns
this to the original Matlab invocation of callR.  This is of course a
simple, artificial example.  The idea is that the R function we call
from Matlab can be quite complicated and can make calls to Matlab
functions during its execution.






\subsection{Example: Optimization}
We now move to a more advanced example in which we use Matlab's
optimization facilities to find the maximum of a mathematical function
defined in R.  We might be interested in doing this if we think
Matlab's optimization methods are more appropriate for the function to
be optimized or if we merely want to validate that we get the same
answer.  We might also want to compare the number of iterations it
takes to reach the optimial value.

For our example, consider a simple log-likelihood function that has
two parameters $\mu$ and $\sigma$ and associated observed data.  Let's
denote this as this as $L(\mu, \sigma\vert X_1, \ldots, X_n)$.  We can
take advantage or R's concept of a closure to combine the observed
data with the particular instance of the function.
Let's use a Normal density function, so our log-likelihood 
can be written in R as
\begin{verbatim}
NormalLikelihood = 
function(x)
{
 n = length(x)
 S2 = sum(x^2)
 S = sum(x)

    # XXX
    # Assume parameters are a single vector, not separate arguments. 
    # Nicer as two arguments. Check Matlab interface to optimization 
    # functionality.
 function(theta) {
    -n*log(sqrt(2*pi*theta[2])) + (S2 - 2*S*theta[1] + theta[1]^2)/theta[2]
 }
}
\end{verbatim}
We also want the derivative, that is the Hessian, of the function.


\begin{verbatim}
data = rnorm(100)
l = NormalLikelihood(data)
ans = .Matlab("optim", l, h, c(0, 1) )
\end{verbatim}


(We expect) The outputs are returned as a single row vector containing
estimates of $\mu$ and $\sigma$.

When we call the Matlab function \MatlabFun{optim} from R,
the Matlab engine takes over control and evaluates that
function call until it terminates (normally or abnormally).

It evaluates the function to be optimized at different points.  Each
time it does this, it passes control back to R which evaluates the
function call with inputs from Matlab.  As this proceeds, data is sent
from Matlab to R and back to Matlab with the result of the evaluation
of the operand at this point in the parameter space.  Eventually, the
Matlab optimizer converges (or not) and terminates and returns the
result to R and yields control back to R.  If we put a simple
statement to produce output in our likelihood function, we can monitor
the different inputs that are passed from Matlab and even display them
in a plot.  (Note that the display will not be updated because of
event loop issues.)


Can this be done? What needs to feasible in our interface and hence in
Matlab?  We need to be able to pass an R function as a "reference" to
Matlab.  We need to be able to identify that object as an R function
so that we can invoke it.  And we need to be able to resolve the R
reference back on the R side when identified in Matlab.  Ideally, we
would be able to extend the Matlab data structures to be able to
represent an R function as a ``callbable'' Matlab value and the
low-level, internal evaluation method would merely be call to the R
function.  If this is not possible, we should be able to dynamically
create a Matlab function within the Matlab session that is callable
and whose body is an invocation of the associated R function:
\begin{verbatim}
f  = function(...)
{
  callR(ref, ...) # get the number of arguments from the call to f.
}
\end{verbatim}
The issue here is how to represent the 'ref' value to 
be able to identify the R function object.

\nocite{*}
\bibliographystyle{amsplain}
\bibliography{rmatlab}

% \thebibliography
%\begin{thebibliography}{99}
%\bibitem
%\end{thebibliography}

\end{document}
